{
  \newmdenv[tikzsetting={draw=black,fill=white,fill opacity=0.7, line width=4pt},backgroundcolor=none,leftmargin=0,rightmargin=0,innertopmargin=4pt,skipbelow=\baselineskip,%
  skipabove=\baselineskip]{TitleBoxEquationalReasoning}

  \usebackgroundtemplate{\includegraphics[width=1.0\paperwidth]{image/title-background.png}}

  \begin{frame}[plain] 
  \title{Equational Reasoning}
  
  \vspace{3em}

  \begin{TitleBoxEquationalReasoning}
    \begin{center}
    {\Large \inserttitle}
    \end{center}
  \end{TitleBoxEquationalReasoning}

  \end{frame}
}

\begin{frame}
\frametitle{Equational Reasoning}
\begin{block}{Equational reasoning is a program property}
that allows replacing arbitrary program expressions with values, ultimately providing a tool to construct non-trivial programs.
\end{block}
\end{frame}

\begin{frame}
\frametitle{Equational Reasoning}
We want \textbf{the potential} to replace expressions with values:
\begin{itemize}
  \item<1> without consequence on program behaviour
  \item<2> for the general case i.e. not having to reason about concretes for abstract concepts \tiny{\emph{cough} \lstinline{scala.Seq} \emph{cough}}\normalsize
\end{itemize}
\end{frame}

\begin{frame}[fragile]
\frametitle{Equational Reasoning}
\begin{lstlisting}[style=scala,mathescape]
  val i: String = $\displaystyle\ldots$
  val result: String = `"abc" concat i`
  method1(`result`, `result`)
  $\displaystyle\ldots$
  method2(`result`)
  $\displaystyle\ldots$
\end{lstlisting}
\begin{tikzpicture}[remember picture,overlay]
\coordinate (aa) at ($(a1)+(4,3.5)$); 
\node[note,draw,callout relative pointer={($(aa)-(5.5,-1.1)$)},right] at (aa) {can this expression replace its value?};
\end{tikzpicture}
\end{frame}


\begin{frame}[fragile]
\frametitle{Equational Reasoning}
\begin{lstlisting}[style=scala,mathescape]
  val i: String = $\displaystyle\ldots$
  // val result: String = "abc" concat i
  method1(`"abc" concat i`, `"abc" concat i`)
  $\displaystyle\ldots$
  method2(`"abc" concat i`)
  $\displaystyle\ldots$
\end{lstlisting}
\begin{tikzpicture}[remember picture,overlay]
\coordinate (aa) at ($(a1)+(4.5,2.5)$);
\node[note,draw,callout relative pointer={($(aa)-(8.0,-2.7)$)},right] at (aa) {has the program changed?};
\node[note,draw,callout relative pointer={($(aa)-(5.0,-2.7)$)},right] at (aa) {has the program changed?};
\node[note,draw,callout relative pointer={($(aa)-(8.0,-1.9)$)},right] at (aa) {has the program changed?};
\end{tikzpicture}
\end{frame}


\begin{frame}[fragile]
\frametitle{Unequational Unreasoning}
\begin{lstlisting}[style=scala,mathescape]
  val i: StringBuilder = $\displaystyle\ldots$
  val b: StringBuilder = "abc"
  val result: StringBuilder = `b append i`
  method1(`result`, `result`)
  $\displaystyle\ldots$
  method2(`result`)
  $\displaystyle\ldots$
\end{lstlisting}
\begin{tikzpicture}[remember picture,overlay]
\coordinate (aa) at ($(a1)+(5.2,7.0)$); 
\node[note,draw,callout relative pointer={($(aa)-(7.0,5.5)$)},right] at (aa) {can this expression replace its value?};
\end{tikzpicture}
\end{frame}


\begin{frame}[fragile]
\frametitle{Unequational Unreasoning}
\begin{lstlisting}[style=scala,mathescape]
  val i: StringBuilder = $\displaystyle\ldots$
  val b: StringBuilder = "abc"
  // val result: StringBuilder = b append i
  method1(`b append i`, `b append i`)
  $\displaystyle\ldots$
  method2(`b append i`)
  $\displaystyle\ldots$
\end{lstlisting}
\begin{tikzpicture}[remember picture,overlay]
\coordinate (aa) at ($(a1)+(4.5,2.5)$);
\node[note,draw,callout relative pointer={($(aa)-(8.7,-2.2)$)},right] at (aa) {has the program changed?};
\node[note,draw,callout relative pointer={($(aa)-(6.5,-2.2)$)},right] at (aa) {has the program changed?};
\node[note,draw,callout relative pointer={($(aa)-(8.7,-1.4)$)},right] at (aa) {has the program changed?};
\end{tikzpicture}
\end{frame}


\begin{frame}
\frametitle{Equational Reasoning}
\begin{center}
The program has changed. Equational reasoning has been lost.
\end{center}
\end{frame}


\begin{frame}
\frametitle{Equational Reasoning}
\begin{block}{Equational reasoning \ldots}
\begin{itemize}
  \item gives rise to a catalogue of practical benefits
  \item is \emph{essential} to a high-performing software development team
\end{itemize}
\end{block}
\end{frame}
