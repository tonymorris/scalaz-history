% To see the difference played out, all you have to do is look at the Scalaz
% mailing list, where we routinely discuss "what is the correct principle here?"
% versus the Scala mailing lists, where the cultural atmosphere is reduced "to
% the sordid, petty senselessness of a bickering family that haggles over
% trivial concretes, while betraying all its major values, selling out its
% future for some spurious advantage of the moment. To make it more grotesque,
% that haggling is accompanied by an aura of hysterical self-righteousness, in
% the form of belligerent assertions that one must compromise with anybody on
% anything (except on the tenet that one must compromise) and by panicky appeals
% to “practicality.”"

\begin{frame}
\frametitle{The ultimate goal}
\begin{itemize}
  \item reason about code to know what it does
  \item incrementally improve code
  \item everything it takes to hit the goal \textbf{and nothing more}
\end{itemize}
\end{frame}


\begin{frame}
\frametitle{The ultimate goal}
\begin{block}{This means we must}
\begin{itemize}
  \item exploit existing ideas that are known to work
  \item efficiently reject ideas that are known to not work
  \item explore new ideas for which there is no known answer
\end{itemize}
\end{block}
\end{frame}


\begin{frame}
\frametitle{Readabillaty Factory}
\begin{block}{To assist in code readability}
\begin{itemize}
  \item we count the number of characters in our source code
  \item we look up the meaning of our identifier names in a dictionary
  \item we use the latest test tools to report the number of pixels in our critical source code
\end{itemize}
\end{block}
\end{frame}

\begin{frame}
\frametitle{Careful analysis}
\begin{center}
Diagram of typical whiteboard session

\includegraphics<1>[height=5.2cm]{image/loljava.png}

Careful selection of identifier names, ensuring a correspondence to the real world.
\end{center}
\end{frame}


\begin{frame}
\frametitle{Reject the dunce cap}
\begin{center}
Just kidding!

\includegraphics<1>[height=5.2cm]{image/loljava-crossed.png}

Crass bullshizzles is rejected from the comfort of an ivory tower.
\end{center}
\end{frame}


\begin{frame}
\frametitle{Principles}
\begin{block}{Scalaz explores three principles}
\begin{enumerate}
  \item \normalsize{\textbf{parametricity} or \textbf{free theorems}}

      \tiny{parametric generalisation to eliminate candidate implementations of a type and construct theorems (documentation) based on type}
  \item \normalsize{\textbf{equational reasoning}}

      \tiny{construct large programs from smaller programs and vice versa}
  \item \normalsize{\textbf{abstraction}}

      \tiny{discard expenditure of repetitious effort}

\end{enumerate}
\end{block}
\end{frame}
